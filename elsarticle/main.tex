%% 
%% Copyright 2007-2024 Elsevier Ltd
%% 
%% This file is part of the 'Elsarticle Bundle'.
%% ---------------------------------------------
%% 
%% It may be distributed under the conditions of the LaTeX Project Public
%% License, either version 1.3 of this license or (at your option) any
%% later version.  The latest version of this license is in
%%    http://www.latex-project.org/lppl.txt
%% and version 1.3 or later is part of all distributions of LaTeX
%% version 1999/12/01 or later.
%% 
%% The list of all files belonging to the 'Elsarticle Bundle' is
%% given in the file `manifest.txt'.
%% 
%% Template article for Elsevier's document class `elsarticle'
%% with harvard style bibliographic references

% \documentclass[preprint,12pt]{elsarticle}

%% Use the option review to obtain double line spacing
%% \documentclass[preprint,review,12pt]{elsarticle}

%% Use the options 1p,twocolumn; 3p; 3p,twocolumn; 5p; or 5p,twocolumn
%% for a journal layout:
%% \documentclass[final,1p,times]{elsarticle}
%% \documentclass[final,1p,times,twocolumn]{elsarticle}
% \documentclass[final,3p,times]{elsarticle}
\documentclass[final,3p,times,twocolumn]{elsarticle}
%% \documentclass[final,5p,times]{elsarticle}
%% \documentclass[final,5p,times,twocolumn]{elsarticle}

%% For including figures, graphicx.sty has been loaded in
%% elsarticle.cls. If you prefer to use the old commands
%% please give \usepackage{epsfig}

%% The amssymb package provides various useful mathematical symbols
\usepackage{amssymb}
%% The amsmath package provides various useful equation environments.
\usepackage{amsmath}
%% The amsthm package provides extended theorem environments
%% \usepackage{amsthm}
\usepackage{hyperref}
\usepackage{fancyhdr}
\usepackage{graphicx}
\usepackage{caption}
\usepackage{subcaption}

% Header setup
\pagestyle{fancy}
\fancyhf{} % clear all header/footer fields
\renewcommand{\headrulewidth}{0pt} % no line under header
\fancyhead[C]{\small Ghosh, S. et~al., SMOPS 2026}
\setlength{\headheight}{14pt} % avoid fancyhdr warning

% Ensure first page (plain) also uses the same header (important with \maketitle)
\fancypagestyle{plain}{%
  \fancyhf{}
  \renewcommand{\headrulewidth}{0pt}
  \fancyhead[C]{\small Ghosh, S. et~al., SMOPS 2026}
}

%% The lineno packages adds line numbers. Start line numbering with
%% \begin{linenumbers}, end it with \end{linenumbers}. Or switch it on
%% for the whole article with \linenumbers.
%% \usepackage{lineno}

\journal{}

\begin{document}

\begin{frontmatter}

%% Title, authors and addresses

%% use the tnoteref command within \title for footnotes;
%% use the tnotetext command for theassociated footnote;
%% use the fnref command within \author or \affiliation for footnotes;
%% use the fntext command for theassociated footnote;
%% use the corref command within \author for corresponding author footnotes;
%% use the cortext command for theassociated footnote;
%% use the ead command for the email address,
%% and the form \ead[url] for the home page:
%% \title{Title\tnoteref{label1}}
%% \tnotetext[label1]{}
%% \author{Name\corref{cor1}\fnref{label2}}
%% \ead{email address}
%% \ead[url]{home page}
%% \fntext[label2]{}
%% \cortext[cor1]{}
%% \affiliation{organization={},
%%             addressline={},
%%             city={},
%%             postcode={},
%%             state={},
%%             country={}}
%% \fntext[label3]{}

\title{Long Term Evolution of Space Debris Clouds in the Cislunar Region} %% Article title

%% use optional labels to link authors explicitly to addresses:
%% \author[label1,label2]{}
%% \affiliation[label1]{organization={},
%%             addressline={},
%%             city={},
%%             postcode={},
%%             state={},
%%             country={}}
%%
%% \affiliation[label2]{organization={},
%%             addressline={},
%%             city={},
%%             postcode={},
%%             state={},
%%             country={}}

\author[UTokyo]{Sourav Ghosh\corref{cor1}}
\ead{sourav.ghosh@space.t.u-tokyo.ac.jp}
\author[ISAE]{ Prathap S. Sharma} 
\author[Ind]{Siddhartha Pundit}
\author[Jain]{Rishikesh Bitla}
\author[VIT]{Subhasmita Prusty}
\author[Padua]{Samarth Badgujar} %% Author name

%% Author affiliation
\affiliation[UTokyo]{organization={The University of Tokyo},%Department and Organization
            %addressline={Bunkyo-ku}, 
            city={Tokyo},
            %postcode={113-0033}, 
            %state={},
            country={Japan}}



\affiliation[ISAE]{organization={ISAE-SUPAERO},%Department and Organization
            %addressline={Bunkyo-ku}, 
            city={Toulouse},
            %postcode={113-0033}, 
            %state={},
            country={France}}

\affiliation[Ind]{organization={Independent Researcher},%Department and Organization
            %addressline={Bunkyo-ku}, 
            city={Bengaluru},
            %postcode={113-0033}, 
            %state={},
            country={India}}

\affiliation[Jain]{organization={Jain (Deemed-to-be University)},%Department and Organization
            %addressline={Bunkyo-ku}, 
            city={Bengaluru},
            %postcode={113-0033}, 
            %state={},
            country={India}}

\affiliation[VIT]{organization={VIT Bhopal University},%Department and Organization
            %addressline={Bunkyo-ku}, 
            city={Bhopal},
            %postcode={113-0033}, 
            %state={},
            country={India}}

\affiliation[Padua]{organization={University of Padua},%Department and Organization
            %addressline={Bunkyo-ku}, 
            city={Padua},
            %postcode={113-0033}, 
            %state={},
            country={Italy}}

\cortext[cor1]{Corresponding author}

%% Abstract
\begin{abstract}
%% Text of abstract
As humanity pushes beyond Earth toward a sustained presence around the Moon, we face a familiar problem in a new neighbourhood: space debris. Decades of activity in near-Earth orbits have already filled space with fragments too small to track reliably yet large enough to threaten spacecraft, and if we repeat that pattern in the vast cislunar region between Earth and the Moon, routine lunar operations could become riskier, costlier, and more fragile. This study looks to understand the complex implications of the formation of debris clouds and their evolution in the cislunar space, and how they affect operations near the Lunar region. We focus on the orbits most likely to matter, Near-Rectilinear Halo Orbits (planned for NASA’s Gateway), Distant Retrograde Orbits, lunar frozen orbits, and Low Lunar Orbit, and simulate debris created by collisions or random breakups, then follow those fragments for months to years. To keep things realistic, we combine the insight of Earth–Moon circular restricted three-body dynamics with higher-fidelity multi-body models that include the Sun’s pull, solar radiation pressure, eclipses, and detailed lunar gravity, so we can capture both the broad transport trends and the messy details. We quantify the chance that debris ultimately strikes the Moon and map where those impacts concentrate over time; we also track how fragments ride natural phase-space “highways,” migrating among orbit families and occasionally threading through a defined keep-out corridor around Gateway. The result is a set of practical outputs, risk maps, survival timelines, and conjunction flux estimates, along with sensitivity to fragment size and reflectivity, all aimed at informing near-term operations. We close with concrete mitigations operators can adopt now, from keep-out zones and station-keeping choices to safer disposal strategies, so today’s small mishaps don’t become tomorrow’s systemic hazard for the Artemis era.
\end{abstract}

%%Graphical abstract
%\begin{graphicalabstract}
%\includegraphics{grabs}
%\end{graphicalabstract}

%%Research highlights
%\begin{highlights}
%\item Research highlight 1
%\item Research highlight 2
%\end{highlights}

%% Keywords
\begin{keyword}
%% keywords here, in the form: keyword \sep keyword
Space Debris \sep Space Situational Awareness \sep Cislunar Space
%% PACS codes here, in the form: \PACS code \sep code

%% MSC codes here, in the form: \MSC code \sep code
%% or \MSC[2008] code \sep code (2000 is the default)

\end{keyword}

\end{frontmatter}

%% Add \usepackage{lineno} before \begin{document} and uncomment 
%% following line to enable line numbers
%% \linenumbers

%% main text
%%

\section{Introduction}
\begin{figure}
    \centering
    \includegraphics[width=0.45\textwidth]{figs/fig1.jpg}
    \caption{Illustration of Space Debris by ESA\cite{esa2019}}\label{fig1}
\end{figure}
The cislunar region, the vast expanse between the Earth and Moon, is emerging as the next frontier for space exploration and development. With growing ambitions from governmental space agencies and private enterprises alike, humanity is preparing to establish a sustained presence beyond low Earth orbit. A manned lunar outpost, once a distant aspiration, now lies within reach. To support such a presence, a wide array of space infrastructure will need to be deployed in the cislunar environment. These include communication relays, navigation systems, surveillance platforms, scientific instruments, and logistics modules, all of which will play crucial roles in enabling long-term human operations on and around the Moon. However, as with any expansion into new orbital regimes, the issue of space debris becomes an inevitable and pressing concern. History has shown us in near-Earth space that operational satellites and platforms are susceptible to malfunctions, fragmentation events, and collisions, all of which generate hazardous debris. In the cislunar region, the problem is compounded by the complex and often chaotic dynamical environment governed by the gravitational interplay between the Earth, Moon, and Sun. These dynamics can cause debris to evolve in unpredictable ways over time, posing significant collision risks not only to other operational assets but also to lunar surface installations and future crewed missions. 

% \begin{figure}
%     \begin{subfigure}{.5\textwidth}
%         \centering
%         \includegraphics[width=0.45\textwidth]{figs/fig1.jpg}
%         \caption{Illustration of Space Debris by ESA\cite{esa2019}}\label{fig1}
%     \end{subfigure}
%     \begin{subfigure}{.5\textwidth}
%         \centering
%         \includegraphics[width=0.45\textwidth]{figs/fig4.png}
%         \caption{NRHO Orbit}\label{fig2}
%     \end{subfigure}
% \end{figure}

\begin{figure}[t]
    \centering
    \includegraphics[width=0.45\textwidth]{figs/fig4.png}
    \caption{NRHO Orbit}\label{fig2}
\end{figure}

This study seeks to address the emerging challenge of debris risk in the cislunar space environment. By modeling and simulating key scenarios, including in-orbit break-up events, high-energy impact trajectories, and long-term orbital evolution, we aim to assess potential debris propagation patterns and their associated hazards. In particular, we investigate how fragments from operational failures or collisions might interact with valuable regions such as Earth-Moon Lagrange points, lunar transfer corridors, lunar orbit, and also near-Earth regions such Geostationary and Geosynchronous orbits. These insights are critical for informing the design of future space traffic management protocols, shielding strategies, and operational policies that will ensure the safety, sustainability, and success of humanity’s next great leap into space. Results presented demonstrate the spread of the debris field over time based on Monte-Carlo simulations of a breakup event in the NRHO, a Distant Retrograde orbit (DRO), and a $\text{L}_2$ Halo Orbit.

\begin{figure}[t]
    \centering
    \includegraphics[width=0.45\textwidth]{figs/fig3x.png}
    \caption{NASA's Lunar Gateway (Artemis Program) \cite{zubrin2023}}\label{fig2a}
\end{figure}

\section{Dynamics Model}
\label{sec2}
\subsection{Circular Restricted Three-Body Problem}
\label{sec2.1}

\begin{figure}[t]
    \centering
    \includegraphics[width=0.45\textwidth]{figs/fig2.png}
    \caption{CR3BP Model}\label{fig3}
\end{figure}

The complex dynamics of the region can be simplified using the Circular Restricted Three-Body Problem (CR3BP). This model makes the assumptions as follows:
\begin{itemize}
    \item The primary and secondary masses (Earth and Moon) are in a circular orbit around the barycenter of the system.
    \item The third body is assumed to be a point mass and exerts no force on the other masses.
\end{itemize}

The system may be described using the following relations \cite{ross2022}.

\begin{equation}
    \ddot x - 2 \dot y = \Omega_x
\end{equation}
\begin{equation}
    \ddot y + 2 \dot x = \Omega_y\\ 
\end{equation}
\begin{equation}
    \ddot z = \Omega_z
\end{equation}

\begin{figure}[t]
    \centering
    \includegraphics[width=0.45\textwidth]{figs/fig3.png}
    \caption{Location of Lagrange Points}\label{fig4}
\end{figure}

where, $\Omega$ is the pseudo-potential given by,
\begin{equation}
    \Omega = \frac{1}{2}(x^2 + y^2) + \frac{1-\mu}{r_1} + \frac{\mu}{r_2}
\end{equation}
\begin{equation}
    r_1 = \sqrt{(x+\mu)^2 + y^2 + z^2}
\end{equation}
\begin{equation}
    r_2 = \sqrt{(x-1+\mu)^2 + y^2 + z^2}
\end{equation}

\subsection{Realms of Possible Motion}
\label{sec2.2}
In the Circular Restricted Three-Body Problem (CR3BP), a spacecraft's movement is restricted to specific spatial realms based on its energy level. This energy state is defined by a conserved value called the Jacobi constant, which determines the boundaries of where the craft can physically travel according to the following expression:
\begin{equation}
    E = 2 \Omega - v^2
\end{equation}
Figure \ref{fig5} illustrates these realms through the use of zero-velocity surfaces (ZVS), which define the physical limits a spacecraft cannot exceed without an injection of energy. Within the rotating reference frame of the Earth–Moon system, the specific shape and reach of these regions are governed by the interplay of gravitational forces and centrifugal potential. Access is strictly governed by the Jacobi constant: motion is only possible in regions where kinetic energy is non-negative, effectively rendering the outside areas "forbidden." In 2D projections, these boundaries appear as zero-velocity curves (ZVCs) that function as dynamic gateways. Depending on the spacecraft's energy level, these gates may open or close, either confining the craft to a local vicinity or permitting transit between different zones, such as moving from Earth’s influence to that of the Moon.

\begin{figure}[t]
    \centering
    \captionsetup{justification=centering}
    \includegraphics[width=0.45\textwidth]{figs/fig5.png}
    \caption{Realms of Possible Motion for different Jacobi Constant values $E$ \cite{ross2022}}\label{fig5}
\end{figure}

\subsection{Propagator Setup}
\label{sec2.3}
A Variable-Step, Variable-Order (VSVO) Adams-Bashforth-Moulton integrator referred to as ODE113 in MATLAB is used to propagate trajectories. 

\section{Breakup Simulations}
\label{sec3}

\subsection{Fragmentation Modeling}
\label{sec3.1}

Spacecraft breakup events may be classified into two main types, collisions and explosions. It is very challenging to accurately model a spacecraft breakup event, since it involves several aspects such as the energy released during the event, the materials of the spacecraft, and several other quantities. Based on observations from events in the near-Earth orbits, NASA developed an empirical model referred to as EVOLVE 4.0. This model gives a good estimate of the number of fragments, their size distribution as a function of the parent satellitess size, fragment area and masses, and also the type of event. 
The characteristic length of any fragment is given by the relation \cite{apetrii2024}:

\begin{equation}
    L_c = \frac{1}{3} (X + Y + Z)
\end{equation}

Based on this, we may obtain the size distribution of fragments from an explosion and collision event respectively given by \cite{apetrii2024}:

\begin{equation}
    N_{\text{frag}}^{\text{exp}} = 6L_c^{-\beta_{\text{exp}}}
\end{equation}
\begin{equation}
    N_{\text{frag}}^{\text{col}} = 0.1 {M}^{0.75} L_c^{-\beta_{\text{col}}}
\end{equation}

where, $\beta_{\text{exp}} = 1.6$ and $\beta_{\text{col}} = 1.71$, and $M$ is the combined masses of the two objects that collided. Subsequently, we may model the area and mass of the fragments using the relations:

\begin{equation}
    m_f = \frac{4}{3} \pi \bigg(\frac{L_c}{2} \bigg)^2 \rho_f 
\end{equation}
\begin{equation}
    A_f = \pi \bigg(\frac{L_c}{2} \bigg)^2
\end{equation}

Above relations may be further simplified to obtain the Area-to-mass ratio of each fragment, given by the relation:
\begin{equation}
    \bigg(\frac{A}{m} \bigg)_f = \frac{3}{4} \frac{1}{L_c \rho_f}
\end{equation}

\subsection{Monte-Carlo Simulations of Breakup events}
\label{sec3.2}
Based on EVOLVE 4.0, Monte-Carlo based simulations are done to generate debris fragments from explosions and collisions, and the spread of fragments post breakup is assumed to be omnidirectional. Hence, a $\Delta v$ profile based on the following is imparted on each fragment:
\begin{equation}
    \Delta v_x = \Delta v \sqrt{1 - u^2} \cos \theta
\end{equation}
\begin{equation}
    \Delta v_x = \Delta v \sqrt{1 - u^2} \sin \theta
\end{equation}
\begin{equation}
    \Delta v_z = \Delta v u
\end{equation}
where, $u$ is a random number uniformly sampled between -1 and 1, and $\theta$ is a random angle sampled between 0 and $2\pi$. The initial mass of the parent satellite is considered to be 1000 kg, and the characteristic lengths of fragments is sampled between 10 cm and 1 m. 1000 runs were performed, and three main orbits were analyzed, which include the Near-Rectilinear Halo Orbit (NRHO), a Distant Retrograde Orbit (DRO), and a $\text{L}_2$ Halo orbit. Each fragment is then propagated over a period of 3, 6, 9, 12, and 60 months to demonstrate their behaviour over a long duration of time. 
The initial positions of each fragment is randomized within a radius of 500 m. 

\subsection{Collisions with the Moon and the Lunar Gateway}
\label{sec3.3}
Upon propagation, fragments observed hitting the Moon or approaching close to the Lunar Gateway in the NRHO are flagged. This helps with assessing the collision risk, and also the potential risks of debris in the region.

\section{Simulations}
\label{sec4}

\subsection{Long Term Propagation}

As stated in section \ref{sec3.2} and \ref{sec3.3}, 1000 Monte-Carlo runs of debris is propagated for each of the orbits studied, and collisions are flagged. A heatmap of fragment location after each time step is generated, which is used to determine the evolution of the debris cloud over a duration of five years. Results are as seen in the following figures.
ADD ALL THE FOOKING PICTURES OF THE HEATMAPS HERE.
%% Add \usepackage{lineno} before \begin{document} and uncomment 
%% following line to enable line numbers
%% \linenumbers

%% main text
%%

%% Use \section commands to start a section
% \section{Example Section}
% \label{sec1}
% %% Labels are used to cross-reference an item using \ref command.

% Section text. See Subsection \ref{subsec1}.

% %% Use \subsection commands to start a subsection.
% \subsection{Example Subsection}
% \label{subsec1}

% Subsection text.

% %% Use \subsubsection, \paragraph, \subparagraph commands to 
% %% start 3rd, 4th and 5th level sections.
% %% Refer following link for more details.
% %% https://en.wikibooks.org/wiki/LaTeX/Document_Structure#Sectioning_commands

% \subsubsection{Mathematics}
% %% Inline mathematics is tagged between $ symbols.
% This is an example for the symbol $\alpha$ tagged as inline mathematics.

% %% Displayed equations can be tagged using various environments. 
% %% Single line equations can be tagged using the equation environment.
% \begin{equation}
% f(x) = (x+a)(x+b)
% \end{equation}

% %% Unnumbered equations are tagged using starred versions of the environment.
% %% amsmath package needs to be loaded for the starred version of equation environment.
% \begin{equation*}
% f(x) = (x+a)(x+b)
% \end{equation*}

% %% align or eqnarray environments can be used for multi line equations.
% %% & is used to mark alignment points in equations.
% %% \\ is used to end a row in a multiline equation.
% \begin{align}
%  f(x) &= (x+a)(x+b) \\
%       &= x^2 + (a+b)x + ab
% \end{align}

% \begin{eqnarray}
%  f(x) &=& (x+a)(x+b) \nonumber\\ %% If equation numbering is not needed for a row use \nonumber.
%       &=& x^2 + (a+b)x + ab
% \end{eqnarray}

% %% Unnumbered versions of align and eqnarray
% \begin{align*}
%  f(x) &= (x+a)(x+b) \\
%       &= x^2 + (a+b)x + ab
% \end{align*}

% \begin{eqnarray*}
%  f(x)&=& (x+a)(x+b) \\
%      &=& x^2 + (a+b)x + ab
% \end{eqnarray*}

%% Refer following link for more details.
%% https://en.wikibooks.org/wiki/LaTeX/Mathematics
%% https://en.wikibooks.org/wiki/LaTeX/Advanced_Mathematics

%% Use a table environment to create tables.
%% Refer following link for more details.
%% https://en.wikibooks.org/wiki/LaTeX/Tables
% \begin{table}[t]%% placement specifier
% %% Use tabular environment to tag the tabular data.
% %% https://en.wikibooks.org/wiki/LaTeX/Tables#The_tabular_environment
% \centering%% For centre alignment of tabular.
% \begin{tabular}{l c r}%% Table column specifiers
% %% Tabular cells are separated by &
%   1 & 2 & 3 \\ %% A tabular row ends with \\
%   4 & 5 & 6 \\
%   7 & 8 & 9 \\
% \end{tabular}
% %% Use \caption command for table caption and label.
% \caption{Table Caption}\label{tab1}
% \end{table}


%% Use figure environment to create figures
%% Refer following link for more details.
%% https://en.wikibooks.org/wiki/LaTeX/Floats,_Figures_and_Captions
% \begin{figure}[t]%% placement specifier
% %% Use \includegraphics command to insert graphic files. Place graphics files in 
% %% working directory.
% \centering%% For centre alignment of image.
% \includegraphics[width=0.5\textwidth]{example-image-a}
% %% Use \caption command for figure caption and label.
% \caption{Figure Caption}\label{fig2}
% %% https://en.wikibooks.org/wiki/LaTeX/Importing_Graphics#Importing_external_graphics
% \end{figure}


%% The Appendices part is started with the command \appendix;
%% appendix sections are then done as normal sections
% \appendix
% \section{Example Appendix Section}
% \label{app1}

% Appendix text.

%% For citations use: 
%%       \citet{<label>} ==> Lamport [21]
%%       \citep{<label>} ==> [21]
%%
% Example citation, See \citet{lamport94}.

%% If you have bib database file and want bibtex to generate the
%% bibitems, please use
%%
%%  \bibliographystyle{elsarticle-num-names} 
%%  \bibliography{<your bibdatabase>}

%% else use the following coding to input the bibitems directly in the
%% TeX file.

%% Refer following link for more details about bibliography and citations.
%% https://en.wikibooks.org/wiki/LaTeX/Bibliography_Management

\begin{thebibliography}{00}

%% For authoryear reference style
%% \bibitem[Author(year)]{label}
%% Text of bibliographic item

% \bibitem[Lamport(1994)]{lamport94}
%   Leslie Lamport,
%   \textit{\LaTeX: a document preparation system},
%   Addison Wesley, Massachusetts,
%   2nd edition,
%   1994.

\bibitem[ESA(2019)]{esa2019}
    European Space Agency: 
    \textit{Recognising Sustainable Behaviour},
    \url{https://www.esa.int/Space_Safety/Space_Debris/Recognising_sustainable_behaviour},
    2019.

\bibitem[Zubrin(2023)]{zubrin2023}
    Zubrin, R.,
    OP-ED | Lunar Gateway or Moon Direct? 
    \url{SpaceNews. https://spacenews.com/op-ed-lunar-gateway-or-moon-direct/},
    2023.

\bibitem[Ross(2022)]{ross2022}
    Koon, W.S., Lo, M.W., Marsden, J.E., Ross, S.D. (2022) 
    Dynamical Systems, the Three-Body Problem and Space Mission Design. 
    Marsden Books, 
    ISBN 978-0-615-24095-4.

\bibitem{apetrii2024}
    Apetrii, M., Celletti, A., Efthymiopoulos, C., Galeş, C., Vartolomei, T. (2024). 
    Simulating a breakup event and propagating the orbits of space debris. Celestial Mechanics and Dynamical Astronomy, 136(5).,
    \url{https://doi.org/10.1007/s10569-024-10205-3}.

\end{thebibliography}
\end{document}

\endinput
%%
%% End of file `elsarticle-template-num-names.tex'.
